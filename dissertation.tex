\documentclass[english,man]{apa6}

\usepackage{amssymb,amsmath}
\usepackage{ifxetex,ifluatex}
\usepackage{fixltx2e} % provides \textsubscript
\ifnum 0\ifxetex 1\fi\ifluatex 1\fi=0 % if pdftex
  \usepackage[T1]{fontenc}
  \usepackage[utf8]{inputenc}
\else % if luatex or xelatex
  \ifxetex
    \usepackage{mathspec}
    \usepackage{xltxtra,xunicode}
  \else
    \usepackage{fontspec}
  \fi
  \defaultfontfeatures{Mapping=tex-text,Scale=MatchLowercase}
  \newcommand{\euro}{€}
\fi
% use upquote if available, for straight quotes in verbatim environments
\IfFileExists{upquote.sty}{\usepackage{upquote}}{}
% use microtype if available
\IfFileExists{microtype.sty}{\usepackage{microtype}}{}

% Table formatting
\usepackage{longtable,booktabs}
\usepackage[counterclockwise]{rotating}   % Landscape page setup for large tables
\usepackage{multirow}		% Table styling
\usepackage{tabularx}		% Control Column width
\usepackage[flushleft]{threeparttable}	% Allows for three part tables with a specified notes section
\usepackage{threeparttablex}            % Lets threeparttable work with longtable
\usepackage{longtable}              % Allows tables to break across pages

\ifxetex
  \usepackage[setpagesize=false, % page size defined by xetex
              unicode=false, % unicode breaks when used with xetex
              xetex]{hyperref}
\else
  \usepackage[unicode=true]{hyperref}
\fi
\hypersetup{breaklinks=true,
            pdfauthor={},
            pdftitle={Perception of Mixed Emotion Across Cultures},
            colorlinks=true,
            citecolor=blue,
            urlcolor=blue,
            linkcolor=black,
            pdfborder={0 0 0}}
\urlstyle{same}  % don't use monospace font for urls

\setlength{\parindent}{0pt}
%\setlength{\parskip}{0pt plus 0pt minus 0pt}

\setlength{\emergencystretch}{3em}  % prevent overfull lines

\setcounter{secnumdepth}{0}
\ifxetex
  \usepackage{polyglossia}
  \setmainlanguage{}
\else
  \usepackage[english]{babel}
\fi

% Manuscript styling
\captionsetup{font=singlespacing,justification=justified}
\usepackage{csquotes}



\usepackage{tikz} % Variable definition to generate author note

% fix for \tightlist problem in pandoc 1.14
\providecommand{\tightlist}{%
  \setlength{\itemsep}{0pt}\setlength{\parskip}{0pt}}

% Essential manuscript parts
  \title{Perception of Mixed Emotion Across Cultures}

  \shorttitle{EMOTION AND CULTURE}


  \author{
          Yay-hyung cho\textsuperscript{1}  }

  \def\affdep{{""}}%
  \def\affcity{{""}}%

  \affiliation{
    \vspace{0.5cm}
          \textsuperscript{1} The University of Michigan  }


%   \def\affinst{{"init", "The University of Michigan"}}%
%   \def\affstate{{"init", ""}}%
%   \def\affcntry{{"init", ""}}%

  \note{
    \vspace{1cm}
    Author note

    \raggedright
    \setlength{\parindent}{0.4in}

    \newcounter{author}

%     %     %       %       \setcounter{author}{0}
%         %           \addtocounter{author}{1}
%         %         \expandafter\edef\csname authorid\endcsname{\theauthor}
%         Yay-hyung cho, \pgfmathparse{\affdep[\authorid]} \pgfmathresult, \pgfmathparse{\affinst[\authorid]} \pgfmathresult, \pgfmathparse{\affcity[\authorid]} \pgfmathresult, \pgfmathparse{\affstate[\authorid]} \pgfmathresult, \pgfmathparse{\affcntry[\authorid]} \pgfmathresult
%       %     .
%     
    Complete departmental affiliations for each author (note the
    indentation, if you start a new paragraph).
    
    Enter author note here.

                      Correspondence concerning this article should be addressed to Yay-hyung cho. E-mail: \href{mailto:choyang@umich.edu}{\nolinkurl{choyang@umich.edu}}
                }

  \abstract{Enter abstract here (note the indentation, if you start a new
paragraph).}
  \keywords{keywords \\

    \indent Word count: X
  }


\begin{document}

\maketitle



{[}1{]} 5.440247e-10 df 111.2358 mean in group american mean in group
japanese 0.367284 1.641304

\begin{verbatim}
\end{verbatim}

\section{Test}\label{test}

0.00

One of the most commonly studied forms of mixed emotions is the
simultaneous experience of positive and negative emotions, often termed
\enquote{dialectical emotions}. Past studies have found that East Asians
experience more mixed emotions than European Americans (Bagozzi, Wong,
\& Yi, 1999; Kitayama, Markus, \& Kurokawa, 2000; Schimmack, Oishi, \&
Diener, 2002). However, do East Asians perceive more mixed emotion when
reading others' emotional expressions? For example, do East Asians
perceive more happiness in frowning faces and more anger in smiling
faces than European Americans?

\section{Mixed Emotion Across
Cultures}\label{mixed-emotion-across-cultures}

There has been some debate regarding whether or not people can
\emph{\enquote{feel}} both positive and negative emotions. Some
researchers argue that people cannot simultaneously feel both emotions
(Green, Goldman, \& Salovey, 1993; Russell, 1980; Russell \& Carroll,
1999), while others maintain that people can feel both emotions at the
same time because positive and negative valences are independent
dimensions (Cacioppo \& Berntson, 1994; Diener \& Iran-Nejad, 1986;
Larsen, McGraw, \& Cacioppo, 2001).

The experience of dialectical emotions has generally been studied in one
of two ways. The first operationalizes mixed emotions in terms of the
magnitude of correlation between positive and negative emotions (Bagozzi
et al, 1999; Miyamoto \& Ryff, 2011; Schmimack et al., 2002), while the
second examines the frequency of co-occurrence of positive and negative
emotions in a given situation (Larsen, McGraw, Mellers, \& Cacioppo,
2004; Miyamoto, Uchida, \& Ellsworth, 2010. Regardless of which method
is used, East Asians reported experiencing more dialectical emotion than
European Americans. The negative correlation between positive and
negative emotions is stronger for European Americans, while the
correlation for East Asians ranges from a weak negative to a positive
relationship (Bagozzi, Wong, \& Yi, 1999; Kitayama, Markus, \& Kurokawa,
2000; Schimmack, Oishi, \& Diener, 2002). The frequency of reporting
opposite-valenced emotions is also higher for East Asians, although this
has only been found for the reporting of negative emotions in positive
situations (Miyamoto, Uchida, \& Ellsworth, 2010 ).

It follows that East Asians should also be more likely to perceive both
positive and negative emotions when inferring another person's
experience of emotion. A number of scholars have commented on the
possibility of \emph{expressing} mixed emotion (Ekman \& Friesen, 1969;
Plutchik, 1962; Tomkins \& McCarter, 1964), but not on the ability to
perceive mixed emotion in the expressions of others, and there been no
cross-cultural research on the perception of mixed emotion. We
hypothesized that East Asians not only experience more mixed emotion,
but also perceive more mixed emotion.

There are a number of mechanisms that can explain the higher emotion
complexity of East Asians compared to European Americans. One
possibility is the tradition ofdialectical thinking in East Asian,
compared to the Western tradition of analytical thinking. Dialectical
thinking refers to the traditional teachings of East Asia about the
complementarity of opposites (i.e.~the ying-yang principle) and the view
that life is full of contradictions and change (Nisbett, Peng, Choi, \&
Norenzayan, 2001; Peng \& Nisbett, 1999). This contrasts with the
Western' analytical way of thinking, which is reflected in linear
thinking (Ji) and a greater focus on the features of an object than its
gestalt (Norenzayan). A number of scholars have suggested that the
greater prevalence of dialectical thinking among East Asians leads them
to perceive positive and negative emotions together more often than
Westerners (Schimmack et al.; Spencer-Rogers et al).

A second possibility involves differences in the construal of the self
(Markus \& Kitayama, 1991). European American socialization contexts
have traditionally embraced independence. The independent self-construal
manifests in the emotional world as a perception of emotions as a
reflection of the authentic self (Uchida, Townsend, Markus, \&
Bergsieker, 2009). East Asian traditions, on the other hand, have
emphasized interdependence resulting in a tendency to see emotions as a
reflection of the person's interactions with other people (Uchida et
al., 2009; Greenfield, 2013; Kashima et al., 1992). A recent analysis by
Grossman et al (2015) showed that interdependence accounted for more
cross-cultural difference than did dialecticism.

Thus, both the dialectical tradition and the interdependent way of
construing the self can lead to the perception of mixed emotion. For
example, when faced with a smiling face, dialectical East Asians can
easily imagine how a person's situation may change for the worse, ,
resulting in the perception of both positive and negative emotions.
Interdependent self-construal can also lead them to perceive
mixed-valence emotion, by causing East Asians to guess multiple external
social factors at the same time to explain a facial expression

\section{General Overview of the Present
Study}\label{general-overview-of-the-present-study}

Our hypothesis is that when perceiving facial expressions, East Asians
recognize more opposite-valence emotions in facial expressions (i.e.,
seeing anger in smiling faces) than European Americans. In order to test
this hypothesis, we created a set of facial expressions varying in the
model's gender, race, expressivity, and valence, and asked Japanese and
European American participants to judge the emotions of the person in
each picture with scales for perception of 13 emotions.

In order to further explore the hypothesis, Study 2 additionally
observed whether participants make internal or external attributions
when judging facial expressions . Our aim was to observe whether the
tendency to make an internal attribution (thinking the person's emotion
expression is due to one's character or personality) or external
attribution (thinking that the person's emotion expression is due to
external factors outside the person) differs across cultures, as well as
to see if this tendency mediates the relationship between culture and
the perception of mixed emotion.

In Study 3, we used an open-ended essay format to investigate the
perception of mixed emotion across cultures. Our understanding of how
emotions are recognized is easily constrained by the methods we use to
ask about that process. Although simpler for the researcher,
closed-ended methods limit the participants' possible answers and
explanations to those the researcher has pre-determined as likely or
appropriate. In contrast, free-response methods allow participants
greater freedom to explain their process of reading facial expressions.
We expected that the Japanese participants would report more opposite
valence emotions than European Americans in these open-ended essays when
asked about their perception of facial expressions.

\section{1 Study 1}\label{study-1}

\section{Method}\label{method}

\subsection{Participants}\label{participants}

Eighty-one European American undergraduate students (36 women and 45
men) at a large Midwestern university (mean age = 18.95, S.D = 1.09,
range from 18 to 23) and 69 Japanese undergraduate students (46 women
and 23 men) from University in Tokyo area (mean = 20.27, S.D = 1.038,
range from 18 to 23), participated in the experiment to fulfill a course
requirement. All of the European American students self-identified as
European Americans who had spent at least the first 18 years of their
lives in the United States.

\subsection{Materials}\label{materials}

\emph{Stimuli for emotion judgment.} We selected photographs of
Caucasian and East Asian, male and female faces from a pre-tested set of
stimuli (Beaupre \& Hess, 2005). Digital images of 16 faces were used in
a 2 x 2 x 2 {[}gender, ethnicity (Caucasian/East Asian), and expression
(smiling or frowning){]} design. There were two different individuals
for each category, and participants did not see the picture of the same
individual twice. For each of the faces, we created a series of stimuli
that gradually morphed from neutral to extreme expressions in ten steps.
All photographs, showing only the head, were presented as
black-and-white passport style pictures.

\emph{Procedure.} In a within-subject study, participants were asked to
answer questions in a paper-and-pencil format in their native language.
Each participant observed one of two sets of stimuli; each set of
stimuli included eight faces, counterbalanced for targeted emotion
display (smiling, frowning), model's race (Asian, White), and model's
gender (male, female). Each eight faces had ten series of photos that
morphed from neutral to extreme joy or extreme anger. Participants were
asked to rate how strongly they thought the model felt each of 10
emotions \emph{(0=Not At All, 1= A Little, 4= Somewhat, to 8=
Extremely): surprise, amusement/enjoyment, contentment/satisfaction,
happiness/pleasure, pride, disgust/hate, fear, contempt/scorn, sadness,
and anger.}

\section{Results and Discussions}\label{results-and-discussions}

\subsection{Perception of Dialectical
Emotions}\label{perception-of-dialectical-emotions}

To test our hypothesis about cultural difference in the perception of
dialectical emotions in facial expressions, we created a variable that
captured the mean number of opposite-valence emotions perceived across
facial stimuli for each participant. We counted intensity ratings of 1
or higher (\enquote{a little} or more) for negative emotions in smiling
faces (e.g., disgust/hate, fear, contempt/scorn, sadness, and anger) and
for positive emotions in frowning faces (e.g.,\emph{amusement/enjoyment,
contentment/satisfaction, happiness/pleasure, and pride}). These counts
were averaged across eight faces. Using a repeated measures MANOVA, we
tested for the between-subject factor of cultural group, controlling for
the effects of stimuli set, and participant gender. Additionally, we
controlled for the within-subject factors of valence of face (smiling
versus frowning), model race (White, Asian), and model gender (female,
male). As predicted (see Figure 1), we found that Japanese participants
(M=1.88, SE=0.12) reported more opposite-valence emotions than European
American participants (M = 0.72, SE = 0.11; \emph{F} (1, 146) = 47.87,
\emph{p} \textless{} .001; partial CI2= .25 ). This finding applied to
both smiling faces (\emph{F}(1,146) =46.38, \emph{p} \textless{}.001)
and frowning faces (\emph{F} (1,146) =33.67, \emph{p} \textless{}.001).

Study 2 was designed to directly replicate. Study 1 while improving the
experimental method by creating more gradation in the series of faces
morphed from neutral to extreme displays. We further improved our
methods by randomly presenting the photographs in a computer-based study
rather than paper-and-pencil study. Due to time constraints, we only
presented female faces and recruited female participants from both
cultural groups.

\section{Study 2}\label{study-2}

\subsection{Participants}\label{participants-1}

Fifty-two female European American (M=19.12, SD=1.00) and 58 female
international Japanese (M=23.17, SD=3.78) students at a large midwestern
University were recruited for the study. As in Study 1, European
American participants self-identified as European American and had lived
in the United States for at least 18 years (average number of years
outside the United States was M=0.67, SD=1.12). Qualified Japanese
participants reported that they had spent no more than five years of
their life outside Japan (average number of years outside Japan was
M=1.28, SD=1.23). All European American participants had two parents and
at least 2 grandparents who were US-born; Japanese participants'
generational background was similar, with 95\% having two Japan-born
parents and 97\% having at least two Japan-born grandparents. European
American students participated in the experiment to receive extra credit
for an introductory psychology course; Japanese participants received
either a 10-dollar gift certificate or extra course credit as
compensation. \#\# Procedure We used the same procedures and photographs
(Beaupre \& Hess, 2005) as in Study 1. For Study 2, however, four faces
rather than eight faces were presented. Faces were morphed from neutral
to extreme emotion displays in twenty steps to create greater nuance
than in Study 1, in which only ten steps were used. Participants
completed a one-hour, computer-based (MediaLab) study in their native
language. In addition to the emotion intensity scales described in Study
1, participants completed a scale of internal and external attributions
of emotions . Otherwise, the procedure was the same as in Study 1.

\subsection{Results and Discussions}\label{results-and-discussions-1}

\emph{Perception of Dialectical Emotions}. To test our hypothesis about
cultural difference in the perception of dialectical emotions in facial
expressions, we created the same variable as in Study 1, which captured
the mean number of opposite-valence emotions perceived across facial
stimuli for each participant. These counts were averaged across the four
faces. Using a repeated measures MANOVA, we tested for the
between-subject factor of cultural group controlling for the
within-subject factors of stimulus set, valence (smiling or frowning),
and model race (White or Asian). Replicating the results from Study 1,
we found that Japanese participants (M=1.39, SE=0.15) reported more
opposite-valence emotions than European American participants (M = 0.92,
SE = 0.16; \emph{F} (1, 107) = 4.88, \emph{p} = .03; partial CI2= .04).
There were no significant effects of stimulus, race of the model, or
emotional display.

In Study 3, we wanted to further explore attributions of emotions to
investigate the reasons for the cultural differences in the perception
of dialectical emotions. To do so, we used an open-ended method in
addition to a closed-ended format, and content-coded participants'
essays.\\Russell and colleagues (1993) refer to Woodworth and Schlosberg
(1954) who describe facial expressions as having emotion labels that
belong to \enquote{broad, overlapping cluster{[}s{]}\ldots{}rather than
specific, discrete basic emotions} (348). Using the emotion terms
described by participants, we created a set of emotion clusters to
better capture participants' descriptions of emotions. To test our
hypotheses in the open-ended emotion data, we created groups of emotion
words that are similar in valence, arousal level, and how they are
communicated in real world. There were three steps in creating the
emotion groups. First, the first author removed subjects' demographic
information from the open-ended responses and coded each emotion word.
Second, the first author created emotion groups out of similar words,
using labels from research on basic emotion (Ekman, 1971) and appraisal
research (Smith, 1985) as a general guide. Third, for each subject we
coded the group as a 1 if they mentioned at least one word in the group
and a 0 if they did not.

\section{Study 3}\label{study-3}

\subsection{Participants}\label{participants-2}

Participants were 304 (172 women) undergraduate students from a large
university from pacific-west and two universities in Japan. 178
identified as European American (112 women), and 83 were either
international students at an University from pacific-west or Japanese
nationals from two universities in Tokyo (60 women). European American
students volunteered in exchange for extra credit for an introduction to
psychology course, and Japanese students participated in exchange for
either a 10-dollar or a 1000-yen gift certificate (equivalent to
10-dollars) as compensation. \#\# Measures and Procedure Participants
from both cultural groups completed a paper-and-pencil survey
individually in their native language. Each participant observed eight
faces varying in the model's race (White, Asian), gender (male, female),
and emotional display (smiling, frowning). The stimulus set was the same
one used in Study 1, 2 (Beaupre \& Hess, 2005). As in Study 2, we used
only female faces. Participants were asked to describe in their own
words what each model was feeling (e.g., \enquote{For each face, please
indicate the feeling(s) of the person in the picture. You may list more
than one feeling. What is Person X feeling?}), as well as their reasons
for their choice of emotion (e.g., \enquote{Why do you think this person
is feeling this way?}) using an open-ended essay format. Lastly,
participants completed a series of demographic questions, including
their gender, age, and number of years they had lived outside the United
States (for European American participants) or Japan (for Japanese
participants). \#\#\#Content Coding of Data The coding team consisted of
five research assistants (Japanese or European American). All coders
were blind to the demographic information. All five coders were trained
to code open-ended data until the value of the inter-rater reliability
Cronbach alpha reached an acceptable level (alpha range from .91 to
1.00).

\emph{Perceived emotions coding}. For emotions, we coded participants'
answers to the question, \enquote{What do you think this person is
feeling?} Responses were coded when the participant attributed a feeling
or motivational state to the model in the picture (i.e., \enquote{She is
happy}); attributions to the situation of the model, however, were not
coded (i.e., \enquote{She is in a great situation}). Table 1 displays
the ten emotion groups that subjects perceived from faces, which we
categorized into three categories: positive (high arousal positive/low
arousal positive/pride/amusement/other positive), negative
(anger/sadness/disgust/contempt/fear/general unpleasantness), and other
emotions (surprise/in thought/unemotional/confusion/neutral). We added
\enquote{general unpleasantness (e.g., unhappy) and `other positive}
(e.g., carefree) emotion groups based on prior research from appraisal
theory of emotion (Smith, 1985) because they did not seem to fit well
with any other emotion group. The final set of emotions, therefore,
included five categories of positive emotions (high arousal positive/low
arousal positive/amusement, pride, other positive), four categories of
other emotions (surprise, in thought/unemotional, confusion, neutral),
and six categories of negative emotions
(anger/sadness/disgust/contempt/fear).

\emph{Internal and external attribution coding}. We also coded
participants' answer to the question \enquote{Why do you think this
person is feeling that way?} Internal attributions were coded as present
if a participant responded that a model was feeling a certain emotion
because of her personality (e.g., \enquote{She is a happy and congenial
person}). External attributions were coded as present when a
participant's response described any situational cause of a model's
emotion. We found two types of situational attributions:
\enquote{Non-social} which referred to an external cause which did not
involve other people (e.g., \enquote{She is frustrated that she cannot
solve a difficult math problem}); and \enquote{Social} which referred to
an external cause involving other people (e.g., \enquote{She looks angry
and stern because she is a teacher disciplining her students} or
\enquote{She is feeling happy and accomplished because she has a good
family resulting in her smiling}).

\section{Results and Discussions}\label{results-and-discussions-2}

\emph{Number of emotion labels.} European Americans (M=1.72, SE=.04)
used more emotion labels than Japanese (M=1.50, SE=.04), t(290)=3.98,
p\textless{}.001. \emph{Target emotion recognition.} Both European
American and Japanese participants reported the targeted emotions
(happiness, anger) more than any other emotion labels (see Table X).
European American participants more frequently listed the target emotion
happiness for smiling displays compared to Japanese participants, chi2
(4, N = 292) = 126.09, p\textless{} .001; a similar trend emerged for
anger mentions in frowning displays, chi2 (4, N = 292) = 8.32, p=.07.
\emph{Perception of Dialectical Emotions.} We computed a percentage
score for each participant of how many opposite-valence emotions they
recognized in a face, divided by the total number of emotions they
observed. For example, for smiling faces, a participant's score would be
the number of negative emotions they listed, divided by sum of the
number of negative, positive, and neutral emotions they listed for that
particular stimulus. Using a repeated measure MANOVA, we tested for the
between-subject factor of cultural group controlling for the effects of
participant gender; we additionally controlled for the within-subject
factors of display type (smiling versus frowning) and model race (White,
Asian). Replicating the results from Studies 1 and 2, we found that
Japanese participants (M=0.20, SE=0.01) reported a greater percentage of
opposite-valence emotions than European American participants (M=0.14,
SE=0.01; F(1, 281) = 17.39, p\textless{} .001; partial CI2= .06). We
also found a greater percentage of dialectical emotions observed in
Asian (M=.29, SE=.01) than White faces (M=.05, SE=.01), F(1, 281)=47.69,
p\textless{}.001).\\\emph{Attribution Style Difference.}European
American participants were more likely to make an internal attribution
than Japanese participants (Table 3). When separately analyzed by the
valence of facial expression, smiling displays produced more internal
attributions than did frowning displays (Table 3). Consistent with
previous literature, Japanese made more external attributions compared
to European Americans for both non-social external attribution and
social external attributions (Table 3 ).

\section{Discussion}\label{discussion}

This research is novel in two ways. First, there is almost no research
investigating the perception of mixed emotion. This cross-cultural
investigation will fill in the gap in the mixed emotion literature.
Secondly, this research used both an open-ended and closed-ended format,
a more flexible method that allows experimenters to explore various
research questions while maintain the same variable across different
formats. However, there remains a crucial question we should consider
for future studies. We do not know which mechanism explains why East
Asians perceive more mixed emotions than European Americans. •
Dialectical style • Interdependent self-construal • The perceived
authenticity of facial expression

\section{References}\label{references}

\setlength{\parindent}{-0.5in} \setlength{\leftskip}{0.5in}
\setlength{\parskip}{8pt}



\end{document}
